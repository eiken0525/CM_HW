% LaTeX Template For MATH 490 @ VCU
\documentclass[11pt]{article}

\usepackage{hyperref}
\usepackage{amsmath}
\usepackage{amsthm}
\usepackage{amssymb}
\usepackage{enumerate}
\usepackage{enumitem}
\usepackage{titlesec}
\usepackage{multicol}
\usepackage{multirow}
\usepackage{mathtools}
\usepackage{mdframed}
\usepackage{tocloft}
\usepackage{tcolorbox}
\usepackage{extarrows}

\setlist{nosep}
\setlist[enumerate]{label=\alph*.}

\renewcommand{\arraystretch}{0.75}

\definecolor{defcolor}{RGB}{255,236,236}    % light red
\definecolor{ngtcolor}{RGB}{255,242,242}    % lighter red
\definecolor{lnkcolor}{RGB}{0,0,180}        % blue
\definecolor{thmcolor}{RGB}{236,236,255}    % light blue
\definecolor{lemcolor}{RGB}{239,239,255}    % lighter blue
\definecolor{procolor}{RGB}{242,242,255}    % lighter lighter blue
\definecolor{crlcolor}{RGB}{245,245,255}    % lighter lighter lighter blue
\definecolor{xmpcolor}{RGB}{255,240,225}    % light orange
\definecolor{rmkcolor}{RGB}{233,255,235}    % light green
\definecolor{axicolor}{RGB}{255,255,233}    % light yellow
\definecolor{notcolor}{RGB}{255,255,244}    % lighter yellow
\definecolor{whacolor}{RGB}{250,250,250}    % lighter gray
\definecolor{reccolor}{RGB}{255,244,255}    % lighter purple

\hypersetup{
    colorlinks,
    citecolor=lnkcolor,
    filecolor=lnkcolor,
    linkcolor=lnkcolor,
    urlcolor=lnkcolor
}

\newtheoremstyle{break}
    {\topsep/1.5} % space above
    {\topsep/2.2} % space below
    {}          % body font
    {}          % indent amount
    {\rmfamily} % theorem head font
    {\textbf{.}}          % punctuation after theorem head
    {0.5em}  % space after theorem head
    {\textbf{\thmname{#1}\thmnumber{ #2}}\thmnote{\text{ (#3)}}}
                % theorem hed spec. (empty = "normal")

\theoremstyle{break}
\newmdtheoremenv{theorem}{Theorem}
\newmdtheoremenv{corollary}[theorem]{Corollary}
\newmdtheoremenv{lemma}[theorem]{Lemma}
\newmdtheoremenv{axiom}[theorem]{Axiom}
\newmdtheoremenv{notation}[theorem]{Notation}
\newmdtheoremenv{definition}[theorem]{Definition}
\newmdtheoremenv{remark}[theorem]{Remark}
\newmdtheoremenv{example}[theorem]{Example}
\newmdtheoremenv{problem}[theorem]{Problem}
\newmdtheoremenv{question}[theorem]{Question}

\DeclareMathOperator{\arcsec}{arcsec}
\DeclareMathOperator{\arccot}{arccot}
\DeclareMathOperator{\arccsc}{arccsc}
\DeclareMathOperator{\interior}{int}
\DeclareMathOperator{\closure}{cl}
\DeclareMathOperator{\boundary}{bd}

\newcommand{\dd}{\text{d}}
\newcommand{\ddi}{\text{$\,$d}}
\newcommand{\qqed}{{\hfill$\blacksquare$}}
\newcommand{\defeq}{\overset{\text{def}}{=}}
\newcommand{\transpose}{\text{T}}

\linespread{1.9}
\setlength{\textwidth}{6.9in}
\setlength{\textheight}{9.2in}
\setlength{\oddsidemargin}{-0.2in}
\setlength{\evensidemargin}{-0.2in}
\setlength{\topmargin}{-0.2in}
\setlength{\headheight}{0in}
\setlength{\headsep}{0in}
\setlength{\footskip}{0.5in}
\setlength{\multicolsep}{6.2pt}
\setlength{\belowdisplayskip}{0pt}
%\setlength{\belowdisplayshortskip}{0pt}
\setlength{\abovedisplayskip}{0pt}
%\setlength{\abovedisplayshortskip}{0pt}

\setcounter{section}{0}
\numberwithin{equation}{theorem}

\makeatletter
\newcommand{\vast}{\bBigg@{4}}
\newcommand{\Vast}{\bBigg@{5}}
\makeatother
\title{Homework 1 of Computational Mathematics}
\author{Chang, Yung-Hsuan\\111652004\\Department of Applied Mathematics}

\begin{document}
\maketitle
\thispagestyle{empty}
\newpage
\pagenumbering{arabic}

\begin{problem}
    Use the intermediate value theorem and Rolle's theorem to show the graph of $f(x)=x^3+2x+k$ crosses the $x$-axis exactly once, regardless of the value of the constant $k$.
\end{problem}
\textbf{Solution}.

\begin{problem}
    Find $\displaystyle\max_{a\leq x\leq b}|f(x)|$ for the following functions and intervals.
    \begin{enumerate}
        \item $f(x)=\dfrac{2-e^x+2x}{3}$, $[0, 1]$
        \item $f(x)=\dfrac{4x-3}{x^2-2x}$, $[0,5, 1]$
        \item $f(x)=2x\cos(2x)-(x-2)^2$, $[2, 4]$
        \item $f(x)=1+e^{-\cos(x-1)}$, $[1, 2]$
    \end{enumerate}
\end{problem}
\textbf{Solution}.

\begin{problem}
    Find the second Taylor polynomial $P_2(x)$ for the function $f(x)=e^x\cos(x)$ about $x_0=0$.
    \begin{enumerate}
        \item Use $P_2(0.5)$ to approximate $f(0.5)$. Find an upper bound for error $|f(0.5)-P_2(0.5)|$ using the error formula, and compare it to the actual error.
        \item Find a bound for the error $|f(0.5)-P_2(0.5)|$ in using $P_2(x)$ to approximate $f(x)$ on the interval $[0, 1]$.
        \item Approximate $\displaystyle\int_{0}^{1}f(x)\ddi x$ using $\displaystyle\int_{0}^{1}P_2(x)\ddi x$.
        \item Find an upper bound for the error in (c) using $\displaystyle\int_{0}^{1}|P_2(x)|\ddi x$.
    \end{enumerate}
\end{problem}
\textbf{Solution}.

\begin{problem}
    Let $f(x)=\dfrac{1}{1-x}$ and $x_0=0$. Find the $n$-th Taylor polynomial $P_n(x)$ for $f(x)$ about $x_0$. Find a value of $n$ necessary for $P_n(x)$ to approximate $f(x)$ to within $10^{-6}$ on $[0, 0.5]$.
\end{problem}
\textbf{Solution}.

\begin{problem}
    Find the largest interval in which $p^\ast$ must lie to approximate $p$ with relative error at most $10^{-4}$ for each value of $p$.
    \begin{enumerate}
        \item $pi$
        \item $e$
        \item $\sqrt{2}$
        \item $\sqrt[3]{7}$
    \end{enumerate}
\end{problem}
\textbf{Solution}.

\begin{problem}
    Let \vspace{-0.6em}
    \begin{equation*}
        f(x)=\dfrac{e^{x}-e^{-x}}{x}. \vspace{-0.6em}
    \end{equation*}
    \begin{enumerate}
        \item Find $\displaystyle\lim_{x\to 0}\dfrac{e^{x}-e^{-x}}{x}$.
        \item Use three-digit rounding arithmatic to evaluate $f(x)$.
        \item Replace each exponential function with its Maclaurin polynomial, and repeat part (b).
        \item The actual value is $f(0.1)=2.003335000$. Find the relative error for the values obtained in part (b) and (c).
    \end{enumerate}
\end{problem}
\textbf{Solution}.



\end{document}