% LaTeX Template For MATH 490 @ VCU
\documentclass[11pt]{article}

\usepackage{hyperref}
\usepackage{amsmath}
\usepackage{amsthm}
\usepackage{amssymb}
\usepackage{enumerate}
\usepackage{enumitem}
\usepackage{titlesec}
\usepackage{multicol}
\usepackage{multirow}
\usepackage{mathtools}
\usepackage{mdframed}
\usepackage{tocloft}
\usepackage{tcolorbox}
\usepackage{extarrows}

\setlist{nosep}
\setlist[enumerate]{label=\alph*.}

\renewcommand{\arraystretch}{0.75}

\definecolor{defcolor}{RGB}{255,236,236}    % light red
\definecolor{ngtcolor}{RGB}{255,242,242}    % lighter red
\definecolor{lnkcolor}{RGB}{0,0,180}        % blue
\definecolor{thmcolor}{RGB}{236,236,255}    % light blue
\definecolor{lemcolor}{RGB}{239,239,255}    % lighter blue
\definecolor{procolor}{RGB}{242,242,255}    % lighter lighter blue
\definecolor{crlcolor}{RGB}{245,245,255}    % lighter lighter lighter blue
\definecolor{xmpcolor}{RGB}{255,240,225}    % light orange
\definecolor{rmkcolor}{RGB}{233,255,235}    % light green
\definecolor{axicolor}{RGB}{255,255,233}    % light yellow
\definecolor{notcolor}{RGB}{255,255,244}    % lighter yellow
\definecolor{whacolor}{RGB}{250,250,250}    % lighter gray
\definecolor{reccolor}{RGB}{255,244,255}    % lighter purple

\hypersetup{
    colorlinks,
    citecolor=lnkcolor,
    filecolor=lnkcolor,
    linkcolor=lnkcolor,
    urlcolor=lnkcolor
}

\newtheoremstyle{break}
    {\topsep/1.5} % space above
    {\topsep/2.2} % space below
    {}          % body font
    {}          % indent amount
    {\rmfamily} % theorem head font
    {.}          % punctuation after theorem head
    {0.5em}  % space after theorem head
    {\textbf{\thmname{#1}\thmnumber{ #2}}\thmnote{\text{ (#3)}}}
                % theorem hed spec. (empty = "normal")

\theoremstyle{break}
\newmdtheoremenv{theorem}{Theorem}
\newmdtheoremenv{corollary}[theorem]{Corollary}
\newmdtheoremenv{lemma}[theorem]{Lemma}
\newmdtheoremenv{axiom}[theorem]{Axiom}
\newmdtheoremenv{notation}[theorem]{Notation}
\newmdtheoremenv{definition}[theorem]{Definition}
\newmdtheoremenv{remark}[theorem]{Remark}
\newmdtheoremenv{example}[theorem]{Example}
\newmdtheoremenv{problem}[theorem]{Problem}
\newmdtheoremenv{question}[theorem]{Question}

\DeclareMathOperator{\arcsec}{arcsec}
\DeclareMathOperator{\arccot}{arccot}
\DeclareMathOperator{\arccsc}{arccsc}
\DeclareMathOperator{\interior}{int}
\DeclareMathOperator{\closure}{cl}
\DeclareMathOperator{\boundary}{bd}

\newcommand{\dd}{\text{d}}
\newcommand{\ddi}{\text{$\,$d}}
\newcommand{\qqed}{{\hfill$\blacksquare$}}
\newcommand{\defeq}{\overset{\text{def}}{=}}
\newcommand{\transpose}{\text{T}}

\linespread{1.9}
\setlength{\textwidth}{6.9in}
\setlength{\textheight}{9.2in}
\setlength{\oddsidemargin}{-0.2in}
\setlength{\evensidemargin}{-0.2in}
\setlength{\topmargin}{-0.2in}
\setlength{\headheight}{0in}
\setlength{\headsep}{0in}
\setlength{\footskip}{0.5in}
\setlength{\multicolsep}{6.2pt}
\setlength{\belowdisplayskip}{0pt}
%\setlength{\belowdisplayshortskip}{0pt}
\setlength{\abovedisplayskip}{0pt}
%\setlength{\abovedisplayshortskip}{0pt}

\setcounter{section}{0}
\numberwithin{equation}{theorem}

\makeatletter
\newcommand{\vast}{\bBigg@{4}}
\newcommand{\Vast}{\bBigg@{5}}
\makeatother
\title{Homework 5 of Computational Mathematics}
\author{Chang, Yung-Hsuan\\111652004\\Department of Applied Mathematics}

\begin{document}
\maketitle
\thispagestyle{empty}
\newpage
\pagenumbering{arabic}

\begin{problem}\label{problem 1} % 
    
\end{problem}
\textbf{Solution}. 


\newpage
\begin{problem}\label{problem 2} % 
    
\end{problem}
\textbf{Solution}. 


\newpage
\begin{problem}\label{problem 3} % 
    
\end{problem}
\textbf{Solution}. 


\newpage
\begin{problem}\label{problem 4} % 
    
\end{problem}
\textbf{Solution}. 


\newpage
\begin{problem}\label{problem 5} % 
    
\end{problem}
\textbf{Solution}. 


\newpage
\begin{problem}\label{problem 6} % 
\end{problem}
\textbf{Solution}. 


\newpage
\begin{problem}\label{problem 7} % 
    
\end{problem}
\textbf{Solution}. 


\newpage
\begin{problem}\label{problem 8} % 
    
\end{problem}
\textbf{Solution}. 


\newpage
\begin{problem}\label{problem 9} % 
    
\end{problem}
\textbf{Solution}. 


\newpage
\begin{problem}\label{problem 10} % 
    
\end{problem}
\textbf{Solution}. 


\newpage
\begin{problem}\label{problem 11} % 
    
\end{problem}
\textbf{Solution}. 


\newpage
\begin{problem}\label{problem 12} % 
    
\end{problem}
\textbf{Solution}. 


\newpage
\begin{problem}\label{problem 13} % 

\end{problem}
\textbf{Solution}. 


\newpage
\begin{problem}\label{problem 14} % 

\end{problem}
\textbf{Solution}. 


\newpage
\begin{problem}\label{problem 15} % 
    
\end{problem}
\textbf{Solution}. 


\newpage
\begin{problem}\label{problem 16} % 
    
\end{problem}
\textbf{Solution}.

\end{document}