% LaTeX Template For MATH 490 @ VCU
\documentclass[11pt]{article}

\usepackage{hyperref}
\usepackage{amsmath}
\usepackage{amsthm}
\usepackage{amssymb}
\usepackage{enumerate}
\usepackage{enumitem}
\usepackage{titlesec}
\usepackage{multicol}
\usepackage{multirow}
\usepackage{mathtools}
\usepackage{mdframed}
\usepackage{tocloft}
\usepackage{tcolorbox}
\usepackage{extarrows}

\setlist{nosep}
\setlist[enumerate]{label=\alph*.}

\renewcommand{\arraystretch}{0.75}

\definecolor{defcolor}{RGB}{255,236,236}    % light red
\definecolor{ngtcolor}{RGB}{255,242,242}    % lighter red
\definecolor{lnkcolor}{RGB}{0,0,180}        % blue
\definecolor{thmcolor}{RGB}{236,236,255}    % light blue
\definecolor{lemcolor}{RGB}{239,239,255}    % lighter blue
\definecolor{procolor}{RGB}{242,242,255}    % lighter lighter blue
\definecolor{crlcolor}{RGB}{245,245,255}    % lighter lighter lighter blue
\definecolor{xmpcolor}{RGB}{255,240,225}    % light orange
\definecolor{rmkcolor}{RGB}{233,255,235}    % light green
\definecolor{axicolor}{RGB}{255,255,233}    % light yellow
\definecolor{notcolor}{RGB}{255,255,244}    % lighter yellow
\definecolor{whacolor}{RGB}{250,250,250}    % lighter gray
\definecolor{reccolor}{RGB}{255,244,255}    % lighter purple

\hypersetup{
    colorlinks,
    citecolor=lnkcolor,
    filecolor=lnkcolor,
    linkcolor=lnkcolor,
    urlcolor=lnkcolor
}

\newtheoremstyle{break}
    {\topsep/1.5} % space above
    {\topsep/2.2} % space below
    {}          % body font
    {}          % indent amount
    {\rmfamily} % theorem head font
    {.}          % punctuation after theorem head
    {0.5em}  % space after theorem head
    {\textbf{\thmname{#1}\thmnumber{ #2}}\thmnote{\text{ (#3)}}}
                % theorem hed spec. (empty = "normal")

\theoremstyle{break}
\newmdtheoremenv{theorem}{Theorem}
\newmdtheoremenv{corollary}[theorem]{Corollary}
\newmdtheoremenv{lemma}[theorem]{Lemma}
\newmdtheoremenv{axiom}[theorem]{Axiom}
\newmdtheoremenv{notation}[theorem]{Notation}
\newmdtheoremenv{definition}[theorem]{Definition}
\newmdtheoremenv{remark}[theorem]{Remark}
\newmdtheoremenv{example}[theorem]{Example}
\newmdtheoremenv{problem}[theorem]{Problem}
\newmdtheoremenv{question}[theorem]{Question}

\DeclareMathOperator{\arcsec}{arcsec}
\DeclareMathOperator{\arccot}{arccot}
\DeclareMathOperator{\arccsc}{arccsc}
\DeclareMathOperator{\interior}{int}
\DeclareMathOperator{\closure}{cl}
\DeclareMathOperator{\boundary}{bd}

\newcommand{\dd}{\text{d}}
\newcommand{\ddi}{\text{$\,$d}}
\newcommand{\qqed}{{\hfill$\blacksquare$}}
\newcommand{\defeq}{\overset{\text{def}}{=}}
\newcommand{\transpose}{\text{T}}

\linespread{1.9}
\setlength{\textwidth}{6.9in}
\setlength{\textheight}{9.2in}
\setlength{\oddsidemargin}{-0.2in}
\setlength{\evensidemargin}{-0.2in}
\setlength{\topmargin}{-0.2in}
\setlength{\headheight}{0in}
\setlength{\headsep}{0in}
\setlength{\footskip}{0.5in}
\setlength{\multicolsep}{6.2pt}
\setlength{\belowdisplayskip}{0pt}
%\setlength{\belowdisplayshortskip}{0pt}
\setlength{\abovedisplayskip}{0pt}
%\setlength{\abovedisplayshortskip}{0pt}

\setcounter{section}{0}
\numberwithin{equation}{theorem}

\makeatletter
\newcommand{\vast}{\bBigg@{4}}
\newcommand{\Vast}{\bBigg@{5}}
\makeatother
\title{Homework 2 of Computational Mathematics}
\author{Chang, Yung-Hsuan\\111652004\\Department of Applied Mathematics}

\begin{document}
\maketitle
\thispagestyle{empty}
\newpage
\pagenumbering{arabic}

\begin{problem}\label{problem 1}
    Use a fixed-point iteration method to determine a solution accurate to within $10^{-2}$ for $x^3-x-1=0$ on $[1, 2]$. Use $p_0=1$.
\end{problem}
\textbf{Solution}. We want to find the solution to $f(x)=x$ with $f(x)=\sqrt{1+\dfrac{1}{x}}$. Then, by calculator,
\begin{align*}
    p_1&=\sqrt{2}=1.414,\\
    p_2&=1.307,\\
    p_3&=1.329,\\
    p_4&=1.324,
\end{align*}
which is accurate to within $10^{-2}$ for $x^3-x-1=0$ on $[1, 2]$. \qed


\newpage
\begin{problem}\label{problem 2}
    Use Theorem 2.1 to find a bound for the number of iterations needed to achieve an approximation with accuracy $10^{-3}$ to the solution of $x^3+x-4=0$ lying in the interval $[1, 4]$. Find an approximation to the root with this degree of accuracy.
\end{problem}
\textbf{Solution}. Let $f(x)=x^3+x-4$. Since $f\in C[1, 4]$ and $f(1)\cdot f(4)=(-2)\cdot 64<0$. The Bisection method generates a sequence $\{p_n\}_{n=1}^\infty$ approaches to a zero $p$ of $f$ with
\begin{equation*}
    |p_n-p|\leq\dfrac{4-1}{2^n},\quad\text{when $n\geq 1$}.
\end{equation*}
Then,
\begin{equation*}
    \dfrac{3}{2^n}\leq10^{-3}\implies n\geq\log_2(3000)>11.
\end{equation*}

\begin{center}
    \includegraphics[width=0.9\textwidth]{problem_2_py.png}
\end{center}
Using the Bisection method, by Python, $p=1.37866$. \qed


\newpage
\begin{problem}\label{problem 3}
    The following four methods are preposed to compute $21^{1/3}$. Rank them in order, based on their apparent speed of convergence, assuming $p_0=1$.
    \begin{enumerate}
        \item $p_n = \dfrac{20p_{n-1} + \frac{21}{{p_{n-1}}^2}}{21}$
        \item $p_n = p_{n-1} - \dfrac{{p_{n-1}}^3-21}{3{p_{n-1}}^2}$
        \item $p_n = p_{n-1} - \dfrac{{p_{n-1}}^4 - 21p_{n-1}}{{p_{n-1}}^2-21}$
        \item $p_n=\sqrt{\dfrac{21}{p_{n-1}}}$
    \end{enumerate}
\end{problem}
\textbf{Solution}.


\newpage
\begin{problem}\label{problem 4}
    Use Theorem 2.3 to show that $g(x)=2^{-x}$ has a unique fixed point on $\left[\dfrac{1}{3}, 1\right]$. Use fixed-point iteration to find an approximation to the fixed point accurate to within $10^{-4}$. Use corollary 2.5 to estimate the number of iterations required to achieve $10^{-4}$ accuracy, and compare this theoratical estimate the the number actually needed.
\end{problem}
\textbf{Solution}. We know that $g\in C\left[\dfrac{1}{3}, 1\right]$ and $g(x)\in[0.5, 0.9637]\subseteq\left[\dfrac{1}{3}, 1\right]$. Then $g$ has at least a fixed point in $[a, b]$. Moreover, $g'(x)$ exists on $\left(\dfrac{1}{3}, 1\right)$. Choose $k=0.7$. Then
\begin{align*}
    \left|\dfrac{\dd}{\dd x}2^{-x}\right|&=\ln2\cdot 2^{-x}\\
    &<\ln2\cdot 2^{-0}\\
    &=\ln2\\
    &<k
\end{align*}
for all $x\in(0, \infty)$. Hence, $g'(x)$ exists on $\left(\dfrac{1}{3}, 1\right)$ and a positive $0<k<1$ exsits with $\left|g'(x)\right|\leq k$ for all $x\in\left(\dfrac{1}{3}, 1\right)$. Then there exists exactly one fixed point in $\left[\dfrac{1}{3}, 1\right]$. It is known the assumption of Theorem 2.4 holds, i.e., $g'(x)$ exists on $\left(\dfrac{1}{3}, 1\right)$ and a positive $0<k<1$ exsits with $\left|g'(x)\right|\leq k$ for all $x\in\left(\dfrac{1}{3}, 1\right)$. By Corollary 2.5, 
\begin{equation*}
    |p_n-p|\leq 0.7^n\max\{0.6-\dfrac{1}{3}, 1-0.6\}.
\end{equation*}
Then,
\begin{equation*}
    0.7^n\max\{0.6-\dfrac{1}{3}, 1-0.6\}\leq 10^{-4} \implies n>23.
\end{equation*}

\begin{center}
    \includegraphics[width=0.9\textwidth]{problem_4_py.png}
\end{center}
By Python, the number of iteration actually needed is 9, which is smaller due to my overestimation for error. \qed


\newpage
\begin{problem}\label{problem 5}
    Let $A$ be a given positive constant and $g(x)=2x-Ax^2$.
    \begin{enumerate}
        \item Show that if fixed-point iteration converges to a nonzero limit, then the limit is $p=\dfrac{1}{A}$, so the inverse of a number can be found using only multiplications and subtractions.
        \item Find an interval about $\dfrac{1}{A}$ for which fixed-point iteration converges, provided $p_0$ is in that interval.
    \end{enumerate}
\end{problem}
\textbf{Solution}.


\newpage
\begin{problem}\label{problem 6}
    Show that if $A$ is any positive number, then the sequence defined by
    \begin{equation*}
        x_n=\dfrac{1}{2}x_{n-1}+\dfrac{A}{2x_{n-1}},\quad\text{for $n\geq 1$},
    \end{equation*}
    converges to $\sqrt{A}$ whenever $x_0>0$.
\end{problem}
\textbf{Solution}.


\newpage
\begin{problem}\label{problem 7}
    Let $f(x)=-x^3-\cos x$. With $p_0=-1$ and $p_1=0$, find $p_3$.
    \begin{enumerate}
        \item Use the Secant method.
        \item Use the method of False Position.
    \end{enumerate}
\end{problem}
\textbf{Solution}.


\newpage
\begin{problem}\label{problem 8}
    Problems involving the amount of money required to pay off a mortgage over a fixed period of time involve the formula
    \begin{equation*}
        A=\dfrac{P}{i}\left(1-(1+i)^{-n}\right),
    \end{equation*}
    known as an \emph{ordinary annuity equation}. In this equation, $A$ is the amount of the mortgage, $P$ is the amount of each payment, and $i$ is the interest rate per period for the $n$ payment periods. Suppose that a 30-year home mortgage in the amount of \$135,000 is needed and that the borrower can afford house payments of at most \$1000 per month. What is the maximal interest rate the borrower can afford to pay?
\end{problem}
\textbf{Solution}.


\newpage
\begin{problem}\label{problem 9}$\ $
    \begin{enumerate}
        \item Show that for any positive integer $k$, the sequence defined by $p_n=\dfrac{1}{n^k}$ converges linearly to $p=0$.
        \item Show that the sequence $p_n=10^{-2^n}$ converges quafratically to $0$.
    \end{enumerate}
\end{problem}
\textbf{Solution}.


\newpage
\begin{problem}\label{problem 10}$\ $
    \begin{enumerate}
        \item The following sequences are linearly convergent. Generate the first five terms of the sequence $\{\hat{p_n}\}$ using Aitken's $\Delta^2$ method.
        \begin{equation*}
            p_0=0.5, \quad p_n=\cos(p_{n-1}), \quad n\geq 1
        \end{equation*}
        \item Use Steffensen's method to find, to an accuracy of $10^{-4}$, the root of $x^3-x-1=0$ that lies in $[1, 2]$.
    \end{enumerate}
\end{problem}
\textbf{Solution}.


\newpage
\begin{problem}\label{problem 11}
    Given a polynomial $P(x) = x^3-5x^2+8x-6$, do the following:
    \begin{enumerate}
        \item Evaluate $P(2)$, $P'(2)$, $P(4)$, and $P'(4)$ by Horner's method.
        \item Find the root of $P(x)$ with error less than $0.00001$ between $[2, 4]$ by using the Newton method with initial point $x_0 = 2$ and $x_0 = 4$. Determin which initial point may lead to the root.
        \item Deflate $P(x)$ into a quadartic ppolynomial by using the results in (b) and find the complex roots of $P(x)$.
        \item Perform one step of Muller's Method starting from initial $(0,P(0))$, $(1,P(1))$ and $(2,P(2))$.
        \item Implement a MATLAB code of Muller's Method to find the complex root within error less than $0.00001$ and compare with the answer you find in (c).
    \end{enumerate}
\end{problem}
\textbf{Solution}.




\end{document}